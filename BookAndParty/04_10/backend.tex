\chapter{Backend}
Per il corretto funzionamento dell'applicazione è fondamentale che l'interfaccia con il DB 
contenga sia i dati degli utenti sia i dati “locali”. particolarmente importante sono i privilegi
che ogni utente ha sul DB. Fondamentale  è la possibilità di interfacciarsi con API di paypal per 
i pagamenti.

\section{Registrazione, modifica dati e pagamento}
\subsection{PayPal}
Utilizzando l’api di paypal dovrà essere possibile eseguire le transazioni tra cliente e locale, 
l’indirizzo e-mail di riferimento per il pagamento sarà quello del locale.

\subsection{SMTP}
Dovrà essere creato un server SMTP per gestire le e-mail inviate in caso di annullamento delle 
prenotazioni, il server SMTP dovrà avere un solo indirizzo e-mail no-replay@bookandparty.it.

\subsection{Calendario}
Oppure dovrà essere prevista nel applicazione l'utilizzo di un api per gestire le prenotazioni 
attraverso Google Calendar.

\section{Communicazione col dispositivo}
L'applicazione dovrà essere in grado di interfacciarsi con il sistema operativo per:
\begin{itemize}
    \item Salvare le credenziali d'accesso.
    \item Verificare se il dispositivo è in light o dark mode.
    \item Gestire richieste per apertura di una pagina web PayPal per effettuare il pagamento. 
\end{itemize}

\section{Utenti}

\subsection{Non registrato}
\begin{itemize}
    \item Dovrà essere in grado di eseguire ricerche e ricevere i dati dal DB.
    \item Deve poter registrare il proprio account, attraverso i dati del form 
    \ref{itm:datiPersonali} che saranno inseriti seguendo la struttura del DB.
    \item Potrà accedere all'account con le proprie credenziali, attraverso un processo di 
    verifica sicuro con il DB.
\end{itemize}

\subsection{Cliente}
\begin{itemize}
    \item Dovrà essere in grado di eseguire ricerche e ricevere i dati pubblici relativi ai 
    locali .
    \item Deve essere in grado di cambiare i dati inseriti al momento della registrazione 
    esclusa \textcolor{red}{l}'e-mail che rimarrà fissa per tutta la durata della vita del account.
    \item Gli sarà possibile usare l’api di PayPal per eseguire pagamenti.
    \item Il cliente deve poter vedere i commenti lasciati sul locale, ma non gli sarà possibile 
    vedere i commenti del locale su di lui, interfacciandosi con il DB.
    \item Il cliente dovrà essere in grado di lasciare commenti sul locale che andranno salvati 
    nel DB.
    \item Sarà possibile recuperare dal DB i dati riguardanti le sue prenotazioni.
    \item Potrà cancellare le prenotazioni come indicato nella \ref{itm:cancPren}
\end{itemize}

\subsection{Locale}
\begin{itemize}
    \item Deve essere in grado di aggiungere tipologie nuove del locale. Tali dati andranno 
    salvati sul DB.
    \item Deve essere in grado di cambiare e/o elimanare i dati sul DB relativi solo alle 
    tipologie create personalmente.
    \item Il locale deve poter vedere le valutazioni (lasciati da altri utenti \textbf{locale} 
    sul cliente prenotante, e gli sarà possibile vedere i commenti del cliente, interfacciandosi 
    con il DB.
    \item Il locale dovrà essere in grado di lasciare una valutazione sul cliente che andrà 
    salvata nel DB. Tale valutazione sarà visibile solo ad altri utenti \textbf{locale}.
    \item Al momento dell'annullamento di una prenotazione, dovrà essere rimborsato il pagamento 
    attraverso l’api di Paypal.
    \item Al momento dell'annullamento di una prenotazione dovrà essere inviata un'email,
    dall'indirizzo no-replay@bookandparty.it, per la communicazione della disdetta.
\end{itemize}

\subsection{Gestore}
\begin{itemize}    
    \item Valgono tutti quelli detti per cliente
    \item Sarà in grado di creare nuovi utenti “locali”, con relativa email e password che 
    verranno aggiunti al DB.
    \item Potrà gestire i locali cambiando tutti i dati sul DB relativi all'accesso (tranne 
    l’e-mail. Vedi \ref{itm:datiLocale}) e sulla descrizione del locale.
\end{itemize}
