\chapter{Requisiti non funzionali}
In questa sezione vengono riportati i requisiti non funzionali del sistema.

\section{Sicurezza}
L'applicazione deve crittografare, tramite l'algoritmo AES, i dati di login dell'u-tente. Tali dati saranno salvati in
un database sicuro.

\section{Privacy}
L'applicazione deve essere conforme al Regolamento europeo per la protezione dei dati (GDPR).

\section{Compatibilita} 
L'applicazione deve essere compatibile sia con iOS 12.0 che Android 7.0 o superiori.

\section{Localizzazione} 
L'applicazione dovrà essere disponibile solo per l'Italia.

\section{Lingua di sistema} 
L'applicazione dovrà essere sia in lingua italiana che in quella inglese.

\section{Scalabilità} \label{sec:scalab}
L'applicazione deve supportare un numero minimo \textcolor{red}{di} mille utenti (registrati e non) simultaneamente e, all'inizio, un 
massimo di registrazioni di 10 mila utenti.

\section{Operatività} 
L'applicazione deve essere sempre disponibile, esclusi problemi dovuti all'infras-truttura dell'applicazione.

\section{Usabilità} 
L'applicazione deve avere un'interfaccia responsiva seguendo le caratteristiche del dispositivo. 
Il design non deve comprendere elementi grafici esterni all'azione eseguibile su quella determinata pagina.
Deve esssere intuitiva per l'utente avvezzo ad altre app. 
Ogni utente dovrà essere in grado di utilizzare l'app nella sua interezza senza il bisogno di una guida e non oltre
i 2 tentativi. 
Il design dell'app si inspirerà al Google Material Theme Design.

\section{Memorizzazione} 
L'applicazione deve poter memorizzare solo i dati di login, ma avrà salvato nel database i dati di
ogni singolo utente.

\section{Prestazioni} \label{sec:prestazioni}
\begin{itemize}
    \item L'applicazione deve essere responsiva, ovvero per ogni azione che l'utente com-pie il tempo di attesa non deve essere
    superiore ai 2 secondi.

    \item Il sistema inoltre, dopo che un \textbf{cliente} ha effettuato una prenotazione, deve essere in grado di aggiornare il
    numero di locali/tipologie disponibili in un tempo massimo 1 secondo.

    \item Nell'evento in cui due utenti guardino lo stesso locale, ed il primo utente prenoti tutti i posti o ne prenoti a 
    sufficienza per cui il secondo utente non possa eseguire la sua prenotazione, sarà necessario che al momento della 
    tentata prenotazione il secondo utente riceva una notifica toast informandolo della non disponibilità. Mentre la notifica 
    è ancora a schermo i dati relativi al locale dovranno essere aggiornati e alla prossima ricerca con gli stessi parametri 
    il locale non dev'essere più visibile.

    \item \textcolor{red}{Nel caso in cui il \textbf{gestore} cancelli il proprio account o quello del locale bisognerà fare il 
    rimborso a tutti quelli che hanno prenotato.}

    \item \textcolor{red}{Nel caso in cui venga modificata o cancellata una tipologia già prenotata il sistema dovrà inviare al cliente una mail per 
    avvisarlo della modifica/cancellazione apportata e gli verrà rimborsato l'intero importo in caso di cancellazione, la cifra in 
    eccesso se il prezzo è diminuito invece se è aumentato il cliente non dovrà pagare ulteriormente.}
\end{itemize} 

\section{Conferma registrazione}
\begin{itemize}
    \item L'utente riceverà una mail con un link per confermare la registrazione all'applicazione.
    \item Il gestore riceverà una lettera tramite posta per confermare la registrazione del locale all'applicazione.
\end{itemize}
