\chapter{Obiettivi di progettazione del app \NameOfTheProject}

\section*{Scopo del documento}
Questo documento ha lo scopo di descrivere e spiegare come l'applicazione mobile \NameOfTheProject~ dovrà essere
implementata. Il documento è diviso in diverse parti: la prima \textbf{obiettivi} del progetto spiegata quale sono
le necessita che questa applicazione andrà a soddisfare con una descrizione esaustiva del suo funzionamento, la 
seconda è una descrizione di tutti i requisiti che l'applicazione deve soddisfare, l'ultima parte riguarda il 
front-end e il back-end

\section{Obiettivi del progetto}
L'applicazione \NameOfTheProject~ sarà compatibile con la più vasta gamma di dispositivi mobili Android e iOS. Il 
funzionamento dell' applicazione prevede questi utenti:
\begin{itemize}
    \item Non registrato 
    \item Registrato: \begin{itemize}
        \item Cliente
        \item Gestore
        \item Locale
    \end{itemize}
\end{itemize}

\section{Utente non registrato}

Dalla prima apertura dell'applicazione l'utente sarà considerato \textbf{utente non registrato} e, se vorrà, potrà
registrarsi, attraverso un form, inserendo i valori della lista \ref{itm:datiPersonali}.
La registrazione poi dovrà essere confermata con un link invato via e-mail e così l'utente diverrà un 
\textbf{utente registrato cliente}.

L'utente, se non vorrà registrarsi, potrà comunque effettuare ricerche, definite in \ref{itm:ricerca}, 
potendo poi visualizare i locali nel dettaglio cliccando sopra uno dei risultati, come spiegato nei requisiti 
\ref{sec:visLocali} e \ref{sec:visLocale}.

\section{Utente registrato}
L'utente registrato si differenzia in 3 categorie:  \textbf{cliente}, \textbf{gestore} e \textbf{locale}.

Tutti gli \textbf{utenti registrati} potranno fare login inserendo username/email e password.

\subsection{Cliente}

Il \textbf{cliente}, come l'\textbf{utente non registrato}, potrà effettuare ricerche visualizzando poi
i locali nel dettaglio.
Dopo la ricerca, potrà effettuare una prenotazione, premendo sul apposito pulsante verrà trasferito ad una finestra
di riepilogo \ref{itm:recapPrenotazione}. 
Una volta eseguita la prenotazione sarà possibile gestirla e visualizare le sue informazioni \ref{itm:prenotazione} 
nella sezione \textbf{prenotazioni}. Tutte le prenotazioni effettuate saranno disponibili in questa sezione. Per 
ciascuna preno-tazione passata è possibile lasciare una recensione e una valutazione da 1 a 10, invece sarà possibile
annullare quelle future. 

Andando sulla sezione \textbf{profilo}, il \textbf{cliente} potrà selezionare l'elemento \textbf{gestisci 
account} per modificare i suoi dati, oppure selezionare \textbf{registra locale} per registrare un locale. Se verrà
sottoscritto un locale l'utente passerà da \textbf{cliente} a \textbf{gestore}
\ref{sec:gesAccountCliente}. 

%Spiegazione del URG
\subsection{Gestore}
Il \textbf{gestore} ha tutte le funzionalità di un \textbf{cliente} con l'aggiunta di poter gestire e registrare 
uno o più locali nelle apposite sezioni.

La registrazione del locale avviene attraverso un form, simile a quello per la registrazione di un utente, con i 
valori della lista \ref{itm:datiLocale}. Sarà possibile aggiungere più di un immagine.

Successivamente riceverà una mail con un link per confermare la registrazione. Una volta confermato gli verrà 
permesso di aggiungere i dati detagliati del locale (vedi \ref{itm:infoLocale})

Il locale da quel momento sarà nel sistema ma non sarà visibile fino alla conferma dell'indirizzo del locale. Per
confermare l'indirizzo, il \textbf{gestore} dovrà attendere fino all'arrivo, tramite posta all'indirizzo
del locale, della lettera contenente un codice verifica alfanumerico. Tale codice dovrà essere inserito nell'apposita
casella che comparirà nella pagina del locale specifico il quale si trova nella sezione \textbf{gestisci locale}

Inoltre il \textbf{gestore} potrà apportare modifiche o cancellare ogni suo singolo locale.

%La modificare e il recupero della password verrano desctie nel dettaglio nei requisiti funcionali.

%Spiegazione del URL
\subsection{Locale}

Il \textbf{locale} è un utente generato dalla registrazione da parte del \textbf{gestore}. Viene
generato cosicchè possa essere utilizzato dai dipendenti del locale stesso, senza che il gestore dia i propri dati
d'accesso ma solo quelli del \textbf{locale}. Tale account ha i permessi per gestire le prenotazioni, ovvero vedere
il numero di prenotazioni correnti con i dati degli utenti prenotati, potrà anche annullarle in qualsiasi momento,
rimborsando completamente il cliente, valutare ogni singolo cliente (dopo il giorno prenotato) e vedere le recensioni
lasciate dai clienti potendole commentare.
Gli sarà possibile aggiungere e modificare tipologie del locale (vedi lista \ref{itm:infoLocale}). La cancellazione
delle tipologie sarà possibile solo per quelle inserite dal \textbf{locale} e non sarà possibile modificare o 
cancellare quelle create dal \textbf{gestore}.

