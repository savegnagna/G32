\chapter{Requisiti funzionali}
In questa sezione vengono riportati i requisiti funzionali del sistema.

\section*{Requisiti generali}
L'applicazione \NameOfTheProject~ dovrà essere un app per smartphone.

\noindent Il sistema deve avere due tipi di utente:
\begin{itemize}
    \item Non registrato
    \item Registrato
\end{itemize}

\section{Requisiti condivisi}
Sia l'utente \textbf{non registrato} che quello \textbf{registrato} hanno i seguenti requisiti:

\subsection{Ricerca}\label{sec:ricerca}

L'utente può effettuare una ricerca inserendo nel form di ricerca:
\begin{itemize}
    \item Luogo (città)
    \item Data della prenotazione
    \item Nr. di persone
    \label{itm:ricerca}
\end{itemize}

\subsection{Visualizzazione dei locali} \label{sec:visLocali}
L'utente, dopo aver effettuato la ricerca può visualizzare i locali con i requisiti scelti e il prezzo di ognuno di
essi. Inoltre potrà applicare i seguenti filtri:
\begin{itemize}
    \item Tipologia/e del locale \footnote{sala, saletta, tavolo}
    \item Quantità per tipologia \footnote{nr. di sale, salette, tavoli}
    \item Nr. massimo di persone consentite
    \item Zona aperta o chiusa
    \item Presenza del servizio di catering
    \item Presenza del dj
    \item Presenza del impianto audio
    \item Range di Prezzo
    \label{itm:filtri}
\end{itemize}

\subsection{Visualizzazione del singolo locale} \label{sec:visLocale}
L'utente può selezionare uno dei locali della lista per vedere in dettaglio la descrizione, le foto, i servizi e le
valutazioni del locale.

\section{Utente non registrato}
L'utente \textbf{non registrato} oltre a poter fare ricerche (\ref{sec:ricerca}), visualizzare locali 
(\ref{sec:visLocali}) e visualizzare nel dettaglio un locale (\ref{sec:visLocale}) potrà registrarsi.

\subsection{Accesso}
L'utente \textbf{non registrato} potrà accedere all'applicazione cliccando sulla scritta \textbf{Continua come ospite}
(vedi \ref{fig:login}). L'utente salterà il login ed accederà come utente anonimo.


\subsection{Registrazione}\label{sec:registrazione}
L'utente \textbf{non registrato} può registrarsi usando il form registrazione inserendo:
\begin{itemize}
    \item Nome
    \item Cognome
    \item Data di nascita
    \item Nr. di telefono
    \item E-mail
    \item Password
    \item Conferma password
    \label{itm:datiPersonali}
\end{itemize}
Finito l'inserimento dei dati dovrà premere il pulsante \textbf{Registrami}

\section{Utente registrato}
Il sistema deve avere tre tipi di utenti registrati:
\begin{itemize}
    \item Cliente
    \item Locale
    \item Gestore
\end{itemize}

\subsection{Login}
L'\textbf{utente registrato} potrà fare login tramite l'apposito form, inserendo e-mail e password, e inviando la 
richiesta premendo il pulsante \textbf{Accedi}.

\subsection{Logout}
L'\textbf{utente registrato} potrà fare logout tramite l'apposito pulsante (vedi \ref{fig:profiloCliente}), che si 
trova nella sezione \textbf{Profilo}, il quale lo indirizzerà nella pagina di login (vedi \ref{fig:login}).


\subsection{Recupero password} \label{sec:recPsw}
Cliccando sulla scritta per il recupero password, il \textbf{cliente} riceverà una e-mail 
(all'indirizzo già inserito), contenente un token da inserire nella finestra che si 
aprirà nel app allo stesso tempo del click sulla scritta per il recupero password (vedi \ref{fig:login}).

\section*{Cliente} \label{sec:cliente}
L'utente \textbf{cliente} oltre a poter fare ricerche (\ref{sec:ricerca}), visualizzare locali 
(\ref{sec:visLocali}) e visualizzare nel dettaglio un locale (\ref{sec:visLocale}) e potrà modificare il proprio
account, effettuare prenotazioni, gestire prenotazioni e recensire il locale.

\subsection{Gestione account}
Il \textbf{cliente} può modificare le seguenti informazioni del proprio account:
\begin{itemize}
    \item Nome
    \item Cognome
    \item Data di nascita
    \item Nr. di telefono
    \item Password
    \label{sec:gesAccountCliente}
\end{itemize}
Finite le modifiche dovrà premere il pulsante \textbf{Salva modifiche}

\subsection{Prenotazione}
Il \textbf{cliente} può effettuare la prenotazione del locale premendo sul pulsante \textbf{Prenota}, cosicchè verrà 
reindirizzato a una pagina che mostra il riepilogo della prenotazione contente i seguenti campi:
\begin{itemize}
    \item Nome del locale
    \item La data prenotata, \emph{selezionata al momento della ricerca}
    \item L'ultimo giorno per disdire la prenotazione
    \item Prezzo
    \label{itm:recapPrenotazione}
\end{itemize}
Il \textbf{cliente}, verificando il riepilogo, può decidere se tornare indietro o proseguire con la prenotazione
premendo il pulsante \textbf{Completa prenotazione} in cui verrà reindirizzato alla pagina di PayPal per il pagamento. 
Terminato il pagamento, il \textbf{cliente} verrà reindirizzato a una pagina della sezione \textbf{prenotazioni}
contenente le informazioni della prenotazione ricenvendo, tramite e-mail, la conferma della prenotazione stessa.

\subsection{Gestione delle prenotazioni}
Il \textbf{cliente} può vedere l'elenco delle prenotazioni effettuate e per ognuna di esse verificare le seguenti 
informazioni:
\begin{itemize}
    \item Nr. di conferma
    \item Nome della struttura
    \item Data prenotata
    \item Dati della struttura \begin{itemize}
        \item Nr. di telefono
        \item E-mail
        \item Indirizzo stradale
    \end{itemize}
    \label{itm:prenotazione}
\end{itemize}
Inoltre, fino al giorno prenotato sarà possibile premere il pulsante \textbf{annulla prenotazione} per disdire la
prenotazione, a condizione che se:
\begin{itemize}
    \item non si è superato il giorno per disdire la prenotazione, i soldi vengono rimborsati
    \item si è superato il giorno per disdire la prenotazione, non si può richiedere il rimborso
    \label{itm:cancPren}
\end{itemize}

\subsection{Valutazione del locale da parte del cliente}
Il \textbf{cliente}, dopo il giorno della prenotatazione, nella sezione prenotazioni selezionando il locale, può lasciare una
recensione e assegnare anche un voto da 1 a 10.

\subsection{Registrazione locale} 
Il \textbf{cliente}, andando nella sezione \textbf{profilo}, potrà premere sul pulsante \textbf{registra locale}
per registrare un locale, inserendo nell'apposito form i seguenti dati del locale:
\begin{itemize}
    \item Nome
    \item Indirizzo
    \item Recapito telefonico
    \item E-mail 
    \item Password
    \item Conferma password
    \label{itm:datiLocale}
\end{itemize}
Finita l'operazione, l'utente \textbf{cliente} diverrà un utente \textbf{gestore}.

\section*{Locale} \label{sec:locale}
L'utente \textbf{locale} dovrà effettuare il login cosicchè potrà gestire funzioni specifiche dell'account (gestione tipologie), gestire
le prenotazioni, valutare i clienti e visualizzare le valutazioni da parte dei clienti potendole commentare.

\subsection{Gestione tipologie} \label{sec:gestTipo}
Il \textbf{locale}, nella sezione profilo, può apportare modifiche, aggiunte o rimozioni di tipologie del locale.
Per ogni aggiunta di tipologia dovrà specifare i seguenti valori:
\begin{itemize}
    \item Quantità
    \item Date disponibili
    \item Nr. massimo di persone
    \item Costo
    \item Zona aperta/chiusa
    \item Foto
    \item Servizi:
    \begin{itemize}
        \item Catering
        \item DJ
        \item Impianto audio
    \end{itemize}
    \item Prezzo per ogni servizio 
    \label{itm:infoLocale}
\end{itemize}
Inoltre, potrà modificare ogni singola tipologia contenente i valori sopraindicati oppure potrà cancellare le singole
tipologie, tranne quelle inserite dal \textbf{gestore}. 

\subsection{Gestione prenotazioni}
Il \textbf{locale} potrà visualizzare ogni singola prenotazione, riportate nella sezione prenotazioni, verificando le
seguenti informazioni:
\begin{itemize}
    \item Numero di conferma
    \item Data prenotata
    \item Nr. di persone
    \item Tipologia
    \item Quantità tipologia
    \item Dati del cliente:
    \begin{itemize}
        \item Nome e cognome
        \item Nr. di telefono
        \item E-mail
        \item Valutazione (vedi \ref{sec:valCliente})
    \end{itemize}
\end{itemize}

\subsection{Valutazione cliente} \label{sec:valCliente}
Il \textbf{locale} potrà assegnare un voto da 1 a 10 ad ogni singolo cliente dopo il giorno della prenotatazione. Tale valutazione
sarà visibile solo agli utenti \textbf{locale}.

\subsection{Recensione locale} \label{sec:valCliente}
Il \textbf{locale} potrà vedere e commentare, nella sezione \textbf{recensioni}, ogni recensione ricevuta da i
\textbf{clienti}.

\section*{Gestore}
Il \textbf{gestore} è un multi-utente, quindi avrà le stesse funzionalità di un \textbf{cliente} (vedi 
\ref{sec:cliente}) e potrà gestire uno o più \textbf{locali} (vedi \ref{sec:locale}). Andando sempre nella sezione profilo,
come per il \textbf{cliente}, il \textbf{gestore} potrà registrare nuovi locali. Oltre ciò, potrà modificare o
cancellare per completo un locale.

\subsection{Gestione locale} 
Il \textbf{gestore}, sempre nella sezione \textbf{gestione locali}, oltre a poter gestire i locali come in
\ref{sec:gestTipo}, potrà apportare le seguenti modifiche al singolo locale:
\begin{itemize}
    \item Nome
    \item Recapito telefonico
    \item Password
\end{itemize}
Il \textbf{gestore} può modificare ed elimanere tutte le tipologie che un locale possiede. Inoltre sarà l'unico a poter
cancellare il singolo locale, premendo sul pulsante \textbf{cancella locale}.